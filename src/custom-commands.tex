% Define some social accounts and contact information formats
\newcommand{\rut}[1]{
    {{\faUser}\hspace{0.5em}#1}
}
\newcommand{\linkedin}[1]{
    \href{https://linkedin.com/in/#1}{\textcolor{black}
    {\faLinkedinIn}\hspace{0.5em}#1}
}
\newcommand{\email}[1]{
    \href{mailto:#1}{\textcolor{black}
    {\faEnvelope[regular]}\hspace{0.5em}#1}
}
\newcommand{\github}[1]{
    \href{https://github.com/#1}{\textcolor{black}
    {\faGithub}\hspace{0.5em}#1}
}
\newcommand{\twitter}[1]{
    \href{https://twitter.com/#1}{\textcolor{black}
    {\faTwitter}\hspace{0.5em}#1}
}
\newcommand{\facebook}[1]{
    \href{https://www.facebook.com/#1}{\textcolor{black}
    {\faFacebookF}\hspace{0.5em}#1}
}
\newcommand{\website}[1]{
    \href{https://#1}{\textcolor{black}
    {\faGlobeAmericas}\hspace{0.5em}#1}
}
\newcommand{\phone}[1]{
    \textcolor{black}{\faPhone*}\hspace{0.5em}#1
}
\newcommand{\nationality}[1]{
    \textcolor{black}{\faFlagUsa}\hspace{0.5em}#1
}
\newcommand{\address}[1]{
    \textcolor{black}{\faMapMarker*}\hspace{0.5em}#1
}


% Define new resizable bullet with default 0.7 size for later use
% Taken from https://tex.stackexchange.com/questions/534192/medium-sized-circle-as-a-bullet
\newcommand{\vbullet}[1][.7]{\mathbin{\ThisStyle{\vcenter{\hbox{%
  \scalebox{#1}{$\SavedStyle\bullet$}}}}}%
}

% Convenience commands
\newcommand{\italicitem}[1]{\item{\textit{#1}}}
\newcommand{\bolditem}[1]{\item{\textbf{#1}}}
\newcommand{\entry}[2]{#1 & #2 \tabularnewline}

% Define the resume header
\newcommand{\resumeheader}[6]{
    % \phantomsection
    % \addcontentsline{toc}{chapter}{\@name}
    \begin{tabularx}{\textwidth}{Xr}{
        \begin{tabular}[c]{l}
            \fontsize{35}{45}\selectfont{\color{primaryColor}{\textbf{\@name}}}
            \ifx\empty\@title\empty\else
                \\ \textit{\small\@title}
            \fi
       \end{tabular}
    } & {
        \begin{tabular}[c]{l@{\hspace{1em}}l}
            \entry{\small#4}{\small#1}
            \entry{\small#5}{\small#2}
            \entry{\small#6}{\small#3}
        \end{tabular}
    }
    \end{tabularx}
}

% Renew section command for resume section
\renewenvironment{section}[1]{
    % \vspace{-0.25em}
    \phantomsection
    \addcontentsline{toc}{section}{#1}
    \medskip
    {\color{textColor}\large\bfseries\MakeUppercase{#1}}
    \medskip
    {\color{primaryColor}{\hrule height 2pt}}
    % \textcolor{primaryColor}{ \rule{1\textwidth}{2pt}} 
    \begin{list}{}{
        \setlength{\leftmargin}{1.5em}
    }
    \item[]
}{
    \end{list}
}

% Renew subsection command for resume subsections
% All arguments are compulsory
% #1: Name, #2: Description, #3: Time Period, #4: Location
\renewenvironment{subsection}[5]{
    \phantomsection
    \addcontentsline{toc}{subsection}{#1}
    \textbf{#1} \hfill \textit{#3} \newline
    {\color{primaryColor}\textbf{#2}} \hfill \textit{#4}
    \smallskip
    \if\relax\detokenize{#5}\relax
        % La lista está vacía, no hacer nada
    \else
        \begin{list}{$\vbullet$}{
        \leftmargin=2em
        }
        \itemsep -0.5em \vspace{-0.75em}
        {#5}
        \end{list}
   
    \fi
    \vspace{0.25em}
}


% Define command for resume subsections with no bullets
% All arguments are compulsory
% #1: Name, #2: Description, #3: Time Period, #4: Location
\newenvironment{subsectionnobullet}[4]{
    \phantomsection
    \addcontentsline{toc}{subsection}{#1}
    \textbf{#1} \hfill \textit{#3} \newline
    {\color{primaryColor}\textbf{#2}} \hfill \textit{#4}
    \smallskip
    \begin{list}{}{
        \leftmargin=0em \itemindent=0em \labelwidth=0em \labelsep=0em
    }
    \itemsep -0.7em \vspace{-0.7em}
}{
    \end{list}
}

% Define new sectiontable command, which makes a section with a table
% All arguments are compulsory
% #1: Name of the section, #2: Body of the section
\newcommand{\sectiontable}[2]{
    \begin{section}{#1}
        \begin{adjustwidth}{0.0in}{0.1in}
            \begin{tabularx}{\linewidth}{@{} >{\bfseries}l @{\hspace{5ex}} X @{}}
                #2
            \end{tabularx}
        \end{adjustwidth}
    \end{section}
}


\newcommand{\cvlistitem}[3]{%
    {\bfseries{#1}}\\
    {\color{primaryColor}\bfseries{#2}}\\
    {\textit{#3}}\\

    \medskip
    \vfill\null
    \columnbreak
}

