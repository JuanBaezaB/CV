\begin{section}{Experiencia profesional}
      
    \begin{subsection}{Software Developer - DevOps Jr}{Xintec}{Enero 2024 -- Presente}{Concepción, Chile}
    {
    \item Colaboré como desarrollador Frontend y DevOps en el desarrollo de un sistema llamado Capacity, utilizando tecnologías como Kubernetes, Docker, SonarQube, Next.js y servicios en la nube de AWS. También empleé herramientas de CI/CD como Azure Pipelines y GitLab CI/CD.
    } 
    \end{subsection}

    \begin{subsection}{Full Stack Developer}{Dirección de innovación UCSC}{2022 -- Diciembre 2023}{Concepción, Chile}
    {
    \item Participé activamente en el desarrollo de software del proyecto OREAS, un sistema existente de gestión de residuos hospitalarios, enfocado en el desarrollo de nuevas funcionalidades, mejoras y mantenimiento continuo.
    }
    \end{subsection}
    
    \begin{subsection}{Soporte Técnico}{Facultad de Ciencias Económicas y Administrativas – UCSC}{Septiembre 2022 -- Diciembre 2022}{Concepción, Chile}
    {
    \item Atendí y resolví incidencias y solicitudes de soporte técnico de los profesores, estudiantes y personal administrativo de la facultad.
    }  
    \end{subsection}
    
    \begin{subsection}{Full Stack Developer}{Drup SpA}{Enero 2022 -- Marzo 2022}{Concepción, Chile}
    {
    \item Trabajé en el equipo de mantenimiento y mejora de Epidemaps, un sistema de gestión visual que georreferencia a los pacientes tratados en los establecimientos de atención primaria.
    \item Colaboré en la implementación de nuevas funcionalidades y mejoras en el sistema existente, centrándome en el desarrollo de un modelo predictivo para determinar la probabilidad de un paciente de pertenecer al programa cardiovascular.
    } 
    \end{subsection}
    
\end{section}